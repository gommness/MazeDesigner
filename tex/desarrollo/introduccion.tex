Para el desarrollo de esta aplicación se ha seguido un modelo incremental iterativo, dividido en 4 etapas con una etapa previa de análisis de requisitos. Cada una de ellas subdividida en una fase de diseño, una fase de codificación, una fase de pruebas y finalmente una fase de integración.

Las cuatro etapas han sido divididas atendiendo a funcionalidad y son:
\begin{enumerate}
	\item \textbf{\textit{Canvas}.} En esta etapa se cubren los requisitos relativos a la componente de la aplicación que el usuario utilizaría para la construcción de los \dfnpl{espacio-jugable}, así como todos los submódulos requeridos.
	\item \textbf{Llaves, habitaciones y persistencia.} En esta etapa se cubren los requisitos relativos a la definición de modelos de llaves, la selección de regiones en el canvas para la declaración de habitaciones, el submódulo del canvas para la instanciación de llaves y puertas y la presistencia de todo lo desarrollado hasta el momento.
	\item \textbf{Interfaz gráfica.} En esta etapa se han integrado todas las componentes hasta ahora desarrolladas y se ha extendido su funcionalidad para la comunicación entre las componentes y la representación de la información de los elementos del diseño.
	\item \textbf{Condiciones y exploración del diseño.} En esta etapa se cubren los requisitos relativos a la definición de las clases que gestionan las condiciones de apertura de las puertas así como la exploración del diseño para la comprobación de su correctitud.
\end{enumerate}