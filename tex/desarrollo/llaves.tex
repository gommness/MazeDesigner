Para el desarrollo del módulo de llaves se ha comenzado definiendo una arquitectura \dfn{modelo-vista-controlador}. El modelo representa las llaves, que serán en sí mismos modelos que podrán instanciarse más adelante. La vista es la interfaz gráfica (ver figura \ref{FIG:DEV:TABLALLAVES}) con la que el usuario puede crear, eliminar y modificar modelos de llaves gracias al controlador, que es una componente lógica.

\begin{figure}{FIG:DEV:TABLALLAVES}{Interfaz utilizada para creación, eliminación y edición de los modelos de llaves.}
	\image{10cm}{}{capturas/captura-menu-llaves}
\end{figure}

Después de esto, se ha desarrollado un módulo similar para los \dfnpl{nivel}. En éste la vista está compuesta por una componente de lista como se puede observar en la figura \ref{SBFIG:DEV:TABLANIVELES} y una componente que forma parte de un \textit{canvas} como se ilustra en la figura \ref{SBFIG:DEV:CANVASNIVELES}.
La componente de lista sirve para editar propiedades de los \dfnpl{nivel}, como es el nombre y si es un nivel que será exportado o no. Mientras que la componente del \textit{canvas} sirve para la declaración de qué regiones del diseño conforman los \dfnpl{nivel}.
Tal y como se ilustra en la figura \ref{FIG:DEV:VISTANIVELES}, se pueden declarar niveles rectangulares que no intersequen de igual forma que en el \textit{canvas} se crean los \dfnpl{espacio-jugable} (descrito en la subseccion \ref{SUBSEC:DEVCANVAS}). También se puede eliminar uno de estos \dfnpl{nivel} haciendo click derecho dentro de él.finalmente se puede seleccionar uno con click izquierdo, quedando resaltado tanto en el \textit{canvas} como en la tabla.

\begin{figure}{FIG:DEV:VISTANIVELES}{la interfaz para interactuar con el controlador de \dfnpl{nivel} se compone de dos partes.}
	\subfigure[SBFIG:DEV:TABLANIVELES]{En esta interfaz, el usuario puede editar los datos de un \dfn{nivel} ya creado.}{\image{5cm}{}{capturas/captura-tabla-niveles}} \quad
	\subfigure[SBFIG:DEV:CANVASNIVELES]{En esta interfaz el usuario puede crear, eliminar y seleccionar \dfnpl{nivel}. Los niveles declarados son los rectángulos de color verde. El nivel seleccionado es el rectángulo resaltado en color amarillo.}{\image{5cm}{}{capturas/captura-room-canvas}}
\end{figure}

Finalmente, se ha integrado un \textit{sub-canvas} que permite colocar instancias de los modelos de las llaves en el diseño, así como instanciar puertas y un token que simboliza la posición inicial del jugador. Junto a esto, se ha implementado una interfaz de texto básica en la consola inferior de la aplicación que muestra la información de las instancias seleccionadas.
Todo esto se puede ver reflejado en las figuras \ref{FIG:DEV:INSTANCIALLAVES} y \ref{FIG:DEV:INSTANCIAOTROS}.

\begin{figure}{FIG:DEV:INSTANCIALLAVES}{Se permite que el usuario instancie modelos de llaves. Como se puede observar, cuando se selecciona una llave se resalta con un rectángulo amarillo.}
	\image{12cm}{}{capturas/captura-instanciacion-llaves}
\end{figure}

\begin{figure}{FIG:DEV:INSTANCIAOTROS}{Se permite que el usuario instancie puertas y el token de comienzo para la posterior exploración.}
	\subfigure[SBFIG:DEV:INSTANCIAPUERTAS]{El usuario puede crear puertas en el diseño, representadas con líneas azules cuyos lados están identificados con las letras A y B.}{\image{12cm}{}{capturas/captura-instanciacion-puerta}} \quad
	\subfigure[SBFIG:DEV:INSTANCIATOKEN]{Se puede colocar el token de comienzo, representado con un círculo azul.}{\image{12cm}{}{capturas/captura-creacion-token}} \quad
\end{figure}


\begin{comment}
En esta etapa se desarrollaron los módulos de Llaves, de Habitaciones atendiendo a las necesidades de las funcionalidades de persistencia y se extiende el módulo del Canvas para que integre la instanciación de llaves, la creación de habitaciones y la colocación de puertas.

El módulo de llaves consta de una clase KeyRepository que se encarga de la gestión de las llaves (objetos de la clase Key) a nivel lógico y de la clase KeyListWidget que es una interfaz gráfica a través de la cual el usuario interactua con un objeto de la clase KeyRepository creando nuevas llaves, eliminando llaves anteriormente creadas y modificando llaves ya existentes.

De manera similar se contruye el módulo de Habitaciones, sin embargo, la creación y eliminación de éstas se hace a través de una interfaz que se implementa de forma parecida al Canvas. Para la edición del nombre de las habitaciones y de una propiedad que determinará si la habitación va a ser exportada o no, se ha desarrollado una interfaz gráfica a parte similar a la de las llaves.

El módulo del Canvas se extiende atendiendo a los dos nuevos módulos creados.
Con una nueva clase (RoomCanvas), que depende del Canvas y del Grid, se permite la definición de habitaciones seleccionando una región del diseño.
Con otra clase (InstanceCanvas) que también depende también del Canvas y del Grid, se permite la instanciación de los modelos de llaves (que se definen con el módulo de llaves) en el diseño, así como la colocación de un token de comienzo y las puertas, que ambos elementos se utilizarán más adelante en el módulo de exploración.

También durante esta etapa, en cada una de las clases creadas y modificadas se implementan módulos para la propia representación en JSON y la interpretación de un objeto JSON para extraer sus datos, logrando así métodos necesarios y suficientes para las funcionalidades de persistencia.
\end{comment}