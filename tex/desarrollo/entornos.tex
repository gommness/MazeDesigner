Para la creación de la aplicación se han empleado distintos entornos de desarrollo.
Para la implementación de la aplicación en si, se ha utilizado \textit{qtCreator}, que es un IDE preparado específicamente para el desarrollo de aplicaciones utilizando C++ y Qt.
Se ha investigado el motor de videojuegos \textit{Game Maker Studio 2} tanto a nivel de usuario como su funcionamiento interno en lo que a persistencia y gestión de los recursos de un proyecto se refiere.
Para la elaboración de esta memoria, se ha utilizado \textit{TeXstudio}, que se trata de un entorno que facilita la escritura de documentos LaTeX.