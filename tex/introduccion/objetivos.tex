El proyecto se centra en el desarrollo de una aplicación de escritorio que sirva como entorno de diseño de niveles de cualquier videojuego que, de una forma u otra, contenga una \dfn{mazmorra}.

La herramienta debe ser capaz de explorar el diseño creado por el usuario para asegurar que éste no presenta errores en el progreso del jugador y debe exportar los diseños a un motor de videojuegos escogido (\textit{GameMaker Studio 2}) para continuar con el desarrollo del título en él.

Los objetivos principales a cumplir por la herramienta son:
\begin{enumerate}
	\item Una interfaz para editar el espacio jugable de forma visual.
	\item Una interfaz para delimitar los niveles del juego que más tarde serán exportados al motor de videojuegos.
	\item Una interfaz para colocar elementos del progreso lógico del jugador (llaves, bloqueos y punto de comienzo).
	\item La exportación del diseño elaborado por el usuario al motor \textit{GameMaker Studio 2} para continuar en éste el desarrollo del videojuego.
	\item La exploración del diseño para la detección de errores lógicos en el progreso del jugador diseñado implícitamente.
\end{enumerate}

Para ello, se han desarrollado los principales módulos de la aplicación como proyectos de Qt\cite{qt1}\cite{qt2} independientes unos de otros en la medida de lo posible, que finalmente son integrados en una interfaz gráfica.