En la actualidad, el género de videojuegos de tipo mazmorra ha adquirido bastante popularidad tanto entre grandes desarrolladoras como en el escenario independiente del desarrollo de videojuegos. Se han creado títulos como \textit{Super Metroid (1994)}, \textit{Cave Story (2004)}, \textit{Ori and the Blind Forest (2015)} y \textit{Hollow Knight (2017)} entre muchos otros, que a día de hoy se consideran referentes e incluso juegos de culto.\newline

Este tipo de juegos, centrados en la exploración y progreso a través de un entorno, presentan ciertas necesidades particulares a la hora de diseñar los espacios de juego al ser éstos un elemento tan importante para la experiencia del jugador.
Además de deber cumplir con los requisitos visuales y estéticos del título en cuestión, los espacios de juego se diseñan de manera que el jugador necesite realizar ciertos logros, como avanzar en la trama u obtener un cierto objeto, para progresar y poder explorar nuevas zonas. De esta manera, los espacios de juego son diseñados para controlar el progreso del jugador, guiándole (sin que éste se de cuenta) hacia una experiencia determinada.\newline
Para esto, se suelen emplear distintos patrones de bloqueo para dejar ciertas áreas inaccesibles hasta cumplir con unos requisitos determinados. Estos bloqueos no son necesariamente bidireccionales, de manera que se le permite al jugador acceder a un área pero regresar a la zona anterior inmediatamente no es necesariamente posible. Éste es uno de los patrones más populares, empleado usualmente a modo de tutorial, pero empleado de forma errónea podría ocasionar que el jugador quedara atrapado en un área sin poder salir de la misma.
Además de esto, lo patrones para el desbloqueo de áreas más extendidos son el de power-ups y el de llaves consumibles. En particular este último también puede plantear situaciones en las que algunas áreas queden permanentemente inaccesibles para el jugador si los espacios de juego están diseñados erróneamente y éste hace una determinada exploración.\newline

Debido los patrones problemáticos antes mencionados, el diseño de espacios de juego se vuelve muy costoso conforme éstos aumentan de tamaño, y más aún si el progreso del juego no es lineal. Por esto se plantea el desarrollo de una herramienta que facilite al diseñador la tarea de crear estos espacios de juego y, una vez acabados, los explore para asegurar el correcto diseño de los mismos, esto es, que el jugador no pueda realizar una exploración que como consecuencia le deje permanentemente atascado o que deje un área inaccesible sin remedio.

La intención de este trabajo de fin de grado es desarrollar tal herramienta para facilitar el desarrollo de videojuegos de tipo mazmorra.