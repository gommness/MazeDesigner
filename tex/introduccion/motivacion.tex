En la actualidad, el género de videojuegos de tipo mazmorra ha adquirido bastante popularidad tanto entre grandes desarrolladoras como en el escenario independiente del desarrollo de videojuegos. Se han creado títulos como \textit{Super Metroid\footnote{Nintendo's page for Super Metroid: \url{https://www.nintendo.es/Juegos/Super-Nintendo/Super-Metroid-279613.html}} (Nintendo, 1994)}, \textit{Cave Story\footnote{Cave Story - Doukutsu Monogatari: \url{https://www.cavestory.org/}} (Pixel, Nicalis, 2004)}, \textit{Ori and the Blind Forest\footnote{Ori And The Blind Forest main page: \url{https://www.orithegame.com/blind-forest/}} (Moon Studios, 2015)} y \textit{Hollow Knight\footnote{Hollow Knight main page: \url{https://hollowknight.com/}} (Team Cherry, 2017)} (ver figura \ref{FIG:INTRO:JUEGOS}) entre muchos otros, que a día de hoy se consideran referentes e incluso juegos de culto.\newline
\begin{figure}{FIG:INTRO:JUEGOS}{Algunos de los videojuegos de tipo mazmorra notables.}
	\subfigure[SBFIG:INTRO:SUPERMETROID]{\textit{Super Metroid (Nintendo, 1994)}}{\image{5cm}{}{imagenes/super-metroid}} \quad
	\subfigure[SBFIG:INTRO:CAVESTORY]{\textit{Cave Story (Pixel, Nicalis, 2004)}}{\image{5cm}{}{imagenes/cave-story}} \quad
	\subfigure[SBFIG:INTRO:ORI]{\textit{Ori and the Blind Forest (Moon Studios, 2015)}}{\image{5cm}{}{imagenes/ori}} \quad
	\subfigure[SBFIG:INTRO:HOLLOWKNIGHT]{\textit{Hollow Knight (Team Cherry, 2017)}}{\image{5cm}{}{imagenes/hollow-knight}}
\end{figure}

Este tipo de juegos, centrados en la exploración y avance a través de un entorno, presentan ciertas necesidades particulares a la hora de diseñar los \dfnpl{espacio-de-juego} al ser éstos un elemento tan importante para la experiencia del jugador.
Además de cumplir con los requisitos visuales y estéticos del título en cuestión, los espacios de juego se diseñan de manera que el jugador necesite realizar ciertos logros, como avanzar en la trama u obtener un cierto objeto, para progresar y poder explorar nuevas zonas. De esta manera, los espacios de juego son diseñados para controlar el progreso del jugador, guiándole (sin que éste se de cuenta) hacia una experiencia determinada.\newline
Para esto, se suelen emplear distintos patrones de bloqueo para dejar ciertas áreas inaccesibles hasta cumplir con unos requisitos determinados.\newline
Estos bloqueos no son necesariamente bidireccionales, de manera que se le permite al jugador acceder a un área, pero regresar a la zona anterior inmediatamente no es necesariamente posible. Éste es uno de los patrones más populares, empleado usualmente a modo de tutorial (como se ilustra en el apéndice \ref{CAP:APTUTORIAL}), pero utilizado de forma errónea podría ocasionar que el jugador quedara atrapado en un área sin poder salir de la misma.\newline
Además de esto, los patrones para el desbloqueo de áreas más extendidos son el de \dfnpl{power-up} y el de llaves comportándose como \dfnpl{objeto-consumible} que el jugador debería recolectar y utilizar para abrir caminos antes bloqueados. En particular este último también puede plantear situaciones en las que algunas áreas queden permanentemente inaccesibles si los espacios de juego están diseñados erróneamente y se exploran de una determinada manera.\newline

Debido los patrones problemáticos antes mencionados, el diseño de espacios de juego se vuelve muy costoso conforme éstos aumentan de tamaño, y más aún si el progreso del juego no es lineal, es decir, si el juego permite que el jugador tome diferentes rutas para progresar. esto se puede observar en la figura \ref{FIG:INTRO:MAPA}. Por esto se plantea el desarrollo de una herramienta que facilite al diseñador la tarea de crear estos espacios de juego y, una vez acabados, los explore para asegurar el correcto diseño de los mismos. Es decir, que el jugador no pueda realizar una exploración que como consecuencia le deje permanentemente atascado o que deje un área inaccesible sin remedio.

\begin{figure}{FIG:INTRO:MAPA}{mapa oficial del videojuego \textit{Super Metroid (Nintendo, 1994)} en el que se reflejan todos los \dfnpl{espacio-jugable}, todos los objetos coleccionables tal y como se indica en la leyenda del mapa. Además, las puertas en el mapa están representadas por un código de colores con el que se identifica el \dfn{power-up} requerido para abrirlas (funcionando así como elementos bloqueadores).}
	\image{15cm}{}{imagenes/super-metroid-map}
\end{figure}

La intención de este trabajo de fin de grado es desarrollar tal herramienta para facilitar el desarrollo de videojuegos de tipo mazmorra.