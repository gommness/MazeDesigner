Esta memoria del Trabajo de Fin de Grado está dividida en 6 capítulos además de esta introducción. Éstos son:

\begin{enumerate}
	\item \textbf{Estado del arte} donde detallaremos tecnología existente y herramientas utilizadas para tareas con cierta similitud a las partes de la aplicación a desarrollar, así como tecnologías utilizadas en el desarrollo de videojuegos que tienen directa relación con el producto que se construirá con esta aplicación. También se discutirá en este apartado ventajas e inconvenientes de cada una de las tecnologías que se mencionen.
	\item \textbf{Análisis de Requisitos} donde expondremos los requisitos mínimos, funcionales y no funcionales, que se han de cumplir por la aplicación, definiendo así el alcance de la misma.
	\item \textbf{Diseño} donde se planteará el diseño de la aplicación, el diagrama de clases y se discutirán las distintas decisiones de diseño tomadas en el desarrollo.
	\item \textbf{Desarrollo} donde se detallará el proceso seguido para el desarrollo, hablando de las metodologías empleadas, de dificultades encontradas y de detalles de la implementación.
	\item \textbf{Integración, pruebas y resultados} donde se discutirá el resultado de la aplicación y se mostrarán las pruebas a las que ésta ha sido sometida para asegurar el correcto funcionamiento y el control de la calidad de la misma.
	\item \textbf{Conclusiones y trabajo futuro} donde se hablará del resultado obtenido, la utilidad de la herramienta desarrollada y de posibles mejoras y ampliaciones que podrían llevarse a cabo en el futuro.
\end{enumerate}