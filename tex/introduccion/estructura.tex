Esta memoria del Trabajo de Fin de Grado está dividida en 6 capítulos además de esta introducción. Éstos son:

\begin{enumerate}
	\item \textbf{Estado del arte} donde se detalla tecnología existente y herramientas utilizadas para tareas con cierta similitud a las partes de la aplicación a desarrollar, así como tecnologías utilizadas en el desarrollo de videojuegos que tienen directa relación con el producto que se construirá en este trabajo de fin de grado. También se discutirá en este apartado ventajas e inconvenientes de cada una de las tecnologías que se mencionen.
	\item \textbf{Análisis de Requisitos} donde se exponen los requisitos mínimos, funcionales y no funcionales, que se han de cumplir por la aplicación, definiendo así el alcance de la misma.
	\item \textbf{Diseño} donde se plantea el diseño de la aplicación, el diagrama de clases y se discutirán las distintas decisiones de diseño tomadas en el desarrollo.
	\item \textbf{Desarrollo} donde se detalla el proceso seguido para el desarrollo, hablando de las metodologías empleadas, de dificultades encontradas y de las soluciones llevadas a cabo.
	\item \textbf{Integración, pruebas y resultados} donde se discute el resultado de la aplicación y se muestran las pruebas a las que ésta ha sido sometida para asegurar el correcto funcionamiento y el control de la calidad de la misma.
	\item \textbf{Conclusiones y trabajo futuro} donde se habla del resultado obtenido, la utilidad de la herramienta desarrollada y de posibles mejoras y ampliaciones que podrían llevarse a cabo en el futuro.
\end{enumerate}