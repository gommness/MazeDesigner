% arara: clean: {files: [tfgtfmthesisuam.aux, tfgtfmthesisuam.idx, tfgtfmthesisuam.ilg, tfgtfmthesisuam.ind, tfgtfmthesisuam.bbl, tfgtfmthesisuam.bcf, tfgtfmthesisuam.blg, tfgtfmthesisuam.run.xml, tfgtfmthesisuam.fdb_latexmk, tfgtfmthesisuam.fls, tfgtfmthesisuam.loe, tfgtfmthesisuam.lof, tfgtfmthesisuam.lol, tfgtfmthesisuam.lot, tfgtfmthesisuam.ltb, tfgtfmthesisuam.out, tfgtfmthesisuam.toc, tfgtfmthesisuam.upa, tfgtfmthesisuam.upb, tfgtfmthesisuam.acn, tfgtfmthesisuam.acr, tfgtfmthesisuam.alg, tfgtfmthesisuam.glg, tfgtfmthesisuam.glo, tfgtfmthesisuam.gls, tfgtfmthesisuam.glsdefs, tfgtfmthesisuam.idx,  tfgtfmthesisuam.ilg, tfgtfmthesisuam.xdy, tfgtfmthesisuam.loa, tfgtfmthesisuam.gnuploterrors , tfgtfmthesisuam.mw, tfgtfmthesisuam.fdb_latexmk ]}
% arara: pdflatex: {shell: yes}
% arara: makeglossaries
% arara: makeindex: {style: tfgtfmthesisuam.ist }
% !arara: bibtex
% arara: pdflatex: {shell: yes}
% arara: pdflatex: {shell: yes}
% arara: clean: {files: [tfgtfmthesisuam.aux, tfgtfmthesisuam.idx, tfgtfmthesisuam.ilg, tfgtfmthesisuam.ind, tfgtfmthesisuam.bbl, tfgtfmthesisuam.bcf, tfgtfmthesisuam.blg, tfgtfmthesisuam.run.xml, tfgtfmthesisuam.fdb_latexmk, tfgtfmthesisuam.fls, tfgtfmthesisuam.loe, tfgtfmthesisuam.lof, tfgtfmthesisuam.lol, tfgtfmthesisuam.lot, tfgtfmthesisuam.ltb, tfgtfmthesisuam.out, tfgtfmthesisuam.toc, tfgtfmthesisuam.upa, tfgtfmthesisuam.upb, tfgtfmthesisuam.acn, tfgtfmthesisuam.acr, tfgtfmthesisuam.alg, tfgtfmthesisuam.glg, tfgtfmthesisuam.glo, tfgtfmthesisuam.gls, tfgtfmthesisuam.glsdefs, tfgtfmthesisuam.idx,  tfgtfmthesisuam.ilg, tfgtfmthesisuam.xdy, tfgtfmthesisuam.loa, tfgtfmthesisuam.gnuploterrors , tfgtfmthesisuam.mw, tfgtfmthesisuam.fdb_latexmk ]}


\documentclass[epsbased,copyright,final,printable,covers,extendedindex,firstnumbered,tfg,gnuplot]{tfgtfmthesisuam}

\advisor{Carlos Aguirre}
\levelin{Ingeniería Informática}
\title{Herramienta de diseño de juegos tipo mazmorra para Gamemaker Studio 2}
\subtitle{}
\author{Javier Gómez}
\privateaddress{C\textbackslash\ Francisco Tomás y Valiente Nº 11}
\copyrightdate{3 de Noviembre de 2017}

\dedication{}
\famouscite{A game isn’t something that can be made by one person alone. Well, it’s possible that a really hard-working game creator could finish a game all alone, but even then he would still need a player. \\Games grow and mature when they’re created, played, and conveyed over and over. \\[0.1em] \begin{flushright}Shigesato Itoi\end{flushright}}
% \prefacefile{inicio/prefacio}
% \ackfile{inicio/agradecimientos}
\resumenfile{inicio/resumen}
\abstractfile{inicio/abstract}

\keywords{Algunas}
\palabrasclave{Otras}

\coverdata
{
  Escuela Politécnica Superior \\
  Universidad Autónoma de Madrid \\
  C\textbackslash Francisco Tomás y Valiente nº 11
}

\bibliographyconfig{tfgtfmthesisuam}

\datadir{data}
\graphicsdir{img}
\logosdir{img}
\codesdir{codes}
\usepackage{fancyvrb}

\begin{document}
\show\newdefinition

\chapter{Introducción\label{CAP:INTRODUCCION}{introduccion/introduccion}}
	\section{Motivación\label{SEC:MOTIVACION}}{introduccion/motivacion}
	\section{Objetivos\label{SEC:OBJETIVOS}}{introduccion/objetivos}
	\section{Estructura\label{SEC:ESTRUCTURA}}{introduccion/estructura}
	
\chapter{Estado del Arte\label{CAP:ESTADODELARTE}}{estadodelarte/introduccion}
	\section{Diseño Gráfico\label{SEC:DISENOGRAFICO}}{estadodelarte/grafico}
	\section{Diseño Lógico\label{SEC:DISENOLOGICO}}{estadodelarte/logico}
	\section{Motores y Entornos\label{SEC:MOTORESYENTORNOS}}{estadodelarte/motoresyentornos}
	\section{Conclusiones\label{SEC:ESTADODELARTECONCLUSIONES}}{estadodelarte/conclusiones}
	
\chapter{Análisis de Requisitos\label{CAP:REQUISITOS}}{requisitos/introduccion}
	\section{Requisitos funcionales\label{SEC:REQUISITOSFUNCIONALES}}{requisitos/funcionales}
		\begin{functional}
		\subsection{Requisitos de persistencia\label{SUBSEC:REQPERSISTENCIA}}{requisitos/persistencia}
		\subsection{Requisitos de contexto\label{SUBSEC:REQCONTEXTO}}{requisitos/contexto}
		\subsection{Requisitos del canvas\label{SUBSEC:REQCANVAS}}{requisitos/canvas}
		\subsection{Requisitos de llaves\label{SUBSEC:REQLLAVES}}{requisitos/llaves}
		\subsection{Requisitos de condiciones de apertura\label{SUBSEC:REQCONDICIONES}}{requisitos/condiciones}
		\subsection{Requisitos de comprobación de la completitud\label{SUBSEC:REQCOMPLETITUD}}{requisitos/completitud}
		\subsection{Requisitos de conexión con Game Maker Studio 2\label{SUBSEC:REQCONEXION}}{requisitos/conexion}
		\end{functional}
	\section{Requisitos no funcionales\label{SEC:REQUISITOSNOFUNCIONALES}}{requisitos/nofuncionales}
	
\chapter{Diseño\label{CAP:DISENO}}{diseno/introduccion}
	\section{MazeDesigner GUI\label{SUBSEC:GUI}}{diseno/mazedesignergui}

\chapter{Desarrollo\label{CAP:DESARROLLO}}{desarrollo/introduccion}
	\section{Canvas\label{SUBSEC:DEVCANVAS}}{desarrollo/canvas}
	\section{Llaves\label{SUBSEC:DEVLLAVES}}{desarrollo/llaves}

\chapter{Integración, pruebas y resultados\label{CAP:INTEGRACIONPRUEBASYRESULTADOS}}{integracionpruebasresultados/introduccion}

\chapter{Conclusiones y trabajo futuro\label{CAP:CONCLUSIONES}}{conclusiones/introduccion}
	\section{Conclusiones\label{SEC:CONCLUSIONES}}{conclusiones/conclusiones}
	\section{Trabajo futuro\label{SEC:TRABAJOFUTURO}}{conclusiones/trabajofuturo}

\end{document}
