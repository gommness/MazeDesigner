% arara: clean: {files: [tfgtfmthesisuam.aux, tfgtfmthesisuam.idx, tfgtfmthesisuam.ilg, tfgtfmthesisuam.ind, tfgtfmthesisuam.bbl, tfgtfmthesisuam.bcf, tfgtfmthesisuam.blg, tfgtfmthesisuam.run.xml, tfgtfmthesisuam.fdb_latexmk, tfgtfmthesisuam.fls, tfgtfmthesisuam.loe, tfgtfmthesisuam.lof, tfgtfmthesisuam.lol, tfgtfmthesisuam.lot, tfgtfmthesisuam.ltb, tfgtfmthesisuam.out, tfgtfmthesisuam.toc, tfgtfmthesisuam.upa, tfgtfmthesisuam.upb, tfgtfmthesisuam.acn, tfgtfmthesisuam.acr, tfgtfmthesisuam.alg, tfgtfmthesisuam.glg, tfgtfmthesisuam.glo, tfgtfmthesisuam.gls, tfgtfmthesisuam.glsdefs, tfgtfmthesisuam.idx,  tfgtfmthesisuam.ilg, tfgtfmthesisuam.xdy, tfgtfmthesisuam.loa, tfgtfmthesisuam.gnuploterrors , tfgtfmthesisuam.mw, tfgtfmthesisuam.fdb_latexmk ]}
% arara: pdflatex: {shell: yes}
% arara: makeglossaries
% arara: makeindex: {style: tfgtfmthesisuam.ist }
% !arara: bibtex
% arara: pdflatex: {shell: yes}
% arara: pdflatex: {shell: yes}
% arara: clean: {files: [tfgtfmthesisuam.aux, tfgtfmthesisuam.idx, tfgtfmthesisuam.ilg, tfgtfmthesisuam.ind, tfgtfmthesisuam.bbl, tfgtfmthesisuam.bcf, tfgtfmthesisuam.blg, tfgtfmthesisuam.run.xml, tfgtfmthesisuam.fdb_latexmk, tfgtfmthesisuam.fls, tfgtfmthesisuam.loe, tfgtfmthesisuam.lof, tfgtfmthesisuam.lol, tfgtfmthesisuam.lot, tfgtfmthesisuam.ltb, tfgtfmthesisuam.out, tfgtfmthesisuam.toc, tfgtfmthesisuam.upa, tfgtfmthesisuam.upb, tfgtfmthesisuam.acn, tfgtfmthesisuam.acr, tfgtfmthesisuam.alg, tfgtfmthesisuam.glg, tfgtfmthesisuam.glo, tfgtfmthesisuam.gls, tfgtfmthesisuam.glsdefs, tfgtfmthesisuam.idx,  tfgtfmthesisuam.ilg, tfgtfmthesisuam.xdy, tfgtfmthesisuam.loa, tfgtfmthesisuam.gnuploterrors , tfgtfmthesisuam.mw, tfgtfmthesisuam.fdb_latexmk ]}


\documentclass[epsbased,copyright,final,printable,covers,extendedindex,firstnumbered,tfg,gnuplot]{tfgtfmthesisuam}

\advisor{Carlos Aguirre}
\levelin{Ingeniería de Textos}
\title{Herramienta de diseño de juegos tipo mazmorra para Gamemaker Studio 2}
\subtitle{}
\author{Javier Gómez}
\privateaddress{C\textbackslash\ Francisco Tomás y Valiente Nº 11}
\copyrightdate{3 de Noviembre de 2017}

\dedication{}
\famouscite{Lo peor es cuando has terminado un capítulo\\y la máquina de escribir no aplaude. \\[0.1em] \begin{flushright}Orson Welles\end{flushright}}
% \prefacefile{inicio/prefacio}
% \ackfile{inicio/agradecimientos}
\resumenfile{inicio/resumen}
\abstractfile{inicio/abstract}

\keywords{Algunas}
\palabrasclave{Otras}

\coverdata
{
  Escuela Politécnica Superior \\
  Universidad Autónoma de Madrid \\
  C\textbackslash Francisco Tomás y Valiente nº 11
}

\bibliographyconfig{tfgtfmthesisuam}

\datadir{data}
\graphicsdir{img}
\logosdir{img}
\codesdir{codes}

\begin{document}

\chapter{Introducción\label{CAP:INTRODUCCION}{introduccion/introduccion}}
	\section{Motivación\label{SEC:MOTIVACION}}{introduccion/motivacion}
	\section{Objetivos\label{SEC:OBJETIVOS}}{introduccion/objetivos}
	\section{Estructura\label{SEC:ESTRUCTURA}}{introduccion/estructura}
	
\chapter{Estado del arte\label{CAP:ESTADODELARTE}}{estadodelarte/introduccion}
	\section{Diseño Gráfico\label{SEC:DISENOGRAFICO}}{estadodelarte/grafico}
	\section{Diseño Lógico\label{SEC:DISENOLOGICO}}{estadodelarte/logico}
	
\chapter{Requisitos\label{CAP:REQUISITOS}}{requisitos/introduccion}
	\section{Requisitos funcionales\label{SEC:REQUISITOSFUNCIONALES}}{requisitos/funcionales}
		\subsection{Requisitos de persistencia\label{SUBSEC:PERSISTENCIA}}{requisitos/persistencia}
		\subsection{Requisitos de contexto\label{SUBSEC:CONTEXTO}}{requisitos/contexto}
		\subsection{Requisitos del canvas\label{SUBSEC:CANVAS}}{requisitos/canvas}
		\subsection{Requisitos de llaves\label{SUBSEC:LLAVES}}{requisitos/llaves}
		\subsection{Requisitos de comprobación de la completitud\label{SUBSEC:COMPLETITUD}}{requisitos/completitud}
		\subsection{Requisitos de conexión con Game Maker Studio 2\label{SUBSEC:CONEXION}}{requisitos/conexion}
	\section{Requisitos no funcionales\label{SEC:REQUISITOSNOFUNCIONALES}}{requisitos/nofuncionales}
	
\chapter{Diseño\label{CAP:DISENO}}{diseno/introduccion}

\chapter{Desarrollo\label{CAP:DESARROLLO}}{desarrollo/introduccion}

\chapter{Integración, pruebas y resultados\label{CAP:INTEGRACIONPRUEBASYRESULTADOS}}{integracionpruebasresultados/introduccion}

\chapter{Conclusiones y trabajo futuro\label{CAP:CONCLUSIONES}}{conclusiones/introduccion}
	\section{Conclusiones\label{SEC:CONCLUSIONES}}{conclusiones/conclusiones}
	\section{Trabajo futuro\label{SEC:TRABAJOFUTURO}}{conclusiones/trabajofuturo}
	
% esta parte del documento ira "comentada" a menos que acabe resultando de utilidad
\begin{comment}
\chapter{Estética\label{CAP:ESTETICA}}{estetica/estetica}
  \section{Tipo de documento\label{SEC:TIPODOC}}{estetica/tipodocumento}
  \section{Gama de colores\label{SEC:GAMASEL}}{estetica/gamacolores}
  \section{Colores\label{SEC:COLORES}}{estetica/colores}
  \section{Uso de los colores\label{SEC:USOCOLORES}}{estetica/usocolores}

\chapter{Estructura\label{CAP:ESTRUCTURA}}{estructura/estructura}
  \section{Título, autor, tutor y otras variables\label{SEC:VARIABLES}}{estructura/titulo}
  \section{Índices\label{SEC:INDICES}}{estructura/indices}
  \section{Copyright, dedicatoria y cita inicial\label{SEC:COPYRIGHT}}{estructura/copyright}
  \section[Prefacio, resumen ...]{Prefacio, resumen, abstract, agradecimientos y palabras clave.\label{SEC:PREFACIO}}{estructura/prefacioresumen}
  \section[Partes, capítulos ...]{Partes, capítulos, apartados, subapartados, subsubapartados, párrafos y subpárrafos\label{SEC:CAPITULOS}}{estructura/capitulos}
  \section[Glosario, acrónimos y definiciones]{Glosario, acrónimos y definiciones\label{SEC:GLOSARIO}}{estructura/glosario}
  \section{Referencias\label{SEC:REFERENCIAS}}{estructura/referencias}
  \section{Bibliografía\label{SEC:BIBLIOGRAFIA}}{estructura/bibliografia}

\chapter{Primeros pasos\label{CAP:PRIMEROSPASOS}}{primpas/primpas}
  \section{Estructurar el documento\label{SEC:ESTRUCTURAR}}{primpas/estructuradoc}
  \section{Enlazar la bibliografía\label{SEC:ENLAZBIBLIOGRAFIA}}{primpas/enlazarbib}

\chapter{Elementos internos\label{CAP:ELEMINT}}{elemint/elemint}
  \section{Figuras\label{SEC:FIGURAS}}{elemint/figuras}
    \subsection{Gráficas\label{SS:GRAFICAS}}{elemint/graficas}
    \subsection{Imágenes\label{SS:INMAGENES}}{elemint/imagenes}
    \subsection{Diagramas de Gantt\label{SS:GANTT}}{elemint/gantt}
  \section{Tablas\label{SEC:TABLAS}}{elemint/tablas}
    \subsection{Presupuestos\label{SS:PRESUPUESTOS}}{elemint/presupuestos}
  \section{Cuadros de texto\label{SEC:CUADROS}}{elemint/cuadros}
  \section{Ecuaciones\label{SEC:ECUACIONES}}{elemint/ecuaciones}
  \section{Código\label{SEC:CODIGO}}{elemint/codigo}
  \section{Algoritmos\label{SEC:ALGORITMOS}}{elemint/algoritmos}
  \section{Listas\label{SEC:LISTAS}}{elemint/listas}
  \section{Referencias internas e hiperenlaces\label{SEC:HIPERENLACES}}{elemint/hiperenlaces}
\chapter{Compilación\label{CAP:COMPILACION}}{varios/compilacion}

\appendix

\chapter{Word\textsuperscript{\textregistered} vs. \LaTeXe\label{CAP:WORDLATEX}}{varios/wordlatex}
\chapter{Instalación\label{CAP:INSTALACION}}{varios/instalacion}
\chapter{Packetes incluidos\label{CAP:PAQUETES}}{varios/paquetes}
\chapter{Resumen de opciones del estilo\label{CAP:OPCIONES}}{varios/opciones}
\chapter{Funciones y entornos\label{CAP:FUNCENT}}{varios/funciones}
\end{comment}

\end{document}
