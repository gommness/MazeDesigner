Con la finalización del desarrollo, se plantean las siguientes tareas que podrían realizarse en el futuro para mejorar o ampliar las funcionalidades de \textit{Maze Designer}:
\begin{enumerate}
	\item \textbf{Exportar a otros motores:} se ha desarrollado la aplicación con \textit{Game Maker Studio 2} como motor de videojuegos al que exportar, sin embargo, debido a que los diseños son genéricos, se podría explorar la posibilidad de exportarlos a otros motores, abarcando un mayor número de usuarios.
	\item \textbf{Incluir un sistema de notas:} a la hora de desarrollar esta \textit{Maze Designer}, no se tuvo en cuenta una funcionalidad que podría ser útil, como es la de poder apuntar notas en cualquier parte del diseño.
	\item \textbf{Mantenimiento:} como todo proyecto software, es inevitable la presencia de bugs en un código. Se delega al trabajo futuro la corrección de estos errores según sean reportados.
	\item \textbf{Realización de pruebas por parte de usuarios:} en el desarrollo de este TFG no se han podido realizar pruebas de la aplicación por parte de usuarios reales más allá del desarrollador. Convendría realizar una serie de pruebas para el control de la calidad del producto, de manera que se obtuviera retroalientación para poder seguir mejorando la aplicación.
	\item \textbf{Mejorar el rendimiento de la aplicación:} debido al tiempo de desarrollo, no se han podido optimizar todos los módulos al máximo, utilizando el potencial que ofrece C++. Una optimización de la herramienta podría significar una mejor experiencia del usuario y, en particular, en las tareas de exploración, podría reducirse el tiempo de cómputo.
\end{enumerate}