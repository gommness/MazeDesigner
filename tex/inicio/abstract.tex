The purpose of this document is the development of a desktop application that eases video game developers the task of level design for any game that, in one way or another, contains a set of interconnected scenes in which the player must meet some determined requirements to move from one scene to another. In other words, what is known as a ``maze'' in the industry of game development.

This application will have a graphical interface in which users will be able to build the playable areas, define which regions of those areas make up the levels, visualize the totality of their design as one big connected structure and instantiate a set of logic objects that will leave an underlying implicit logical structure in the design.

The tool created will be able to be integrated in the development process. Exporting the designs built by the users into a video game engine in which the developers will be able to continue working on their product.

Finally, this application will use graph theory's results and some techniques from the field of artificial intelligence to explore the designs created by the users in order to ensure that they are correct. The criteria for correction will make sure that there does not exist any succession of decisions that, as consecuence, have the player not being able to finish the game or leave some areas of it permanently unreachable.

In this document, the current tools available, the motivation behind the development of this tool, the definition of the project itself, its design as a software product, its development, the decisions taken during all the stages of development, the conclusions, the results and potential future upgrades or updates will be discussed.


\keywords{desktop application, video game, design, graphical interface, playable areas, levels, instantiate, logic objects, implicit logical structure, video game engine, graph theory, artificial intelligence}
