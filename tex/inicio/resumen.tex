El objeto de este trabajo de fin de grado es el desarrollo de una aplicación de escritorio que facilite a los desarrolladores de videojuegos la tarea del diseño de niveles para cualquier videojuego que de una forma u otra contengan lo que en la industria se denomina como ``mazmorra''.

Esta aplicación tendrá una interfaz gráfica en la que los usuarios podrán construir los espacios jugables, definir qué regiones de los mismos conforman los niveles, visualizar la totalidad de su diseño como una gran estructura conjunta y crear instancias de una serie de objetos lógicos para dejar una estructura lógica implícita en el diseño.

Además, la herramienta creada deberá poder ser integrada en el proceso de desarrollo. Exportando los diseños construidos por el usuario a un motor de videojuegos donde los desarrolladores podrán continuar con el trabajo en su producto.

Finalmente, esta aplicación utilizará resultados de la teoría de grafos y algunas técnicas del ámbito de la inteligencia artificial para explorar los diseños creados por los usuarios y así comprobar que son correctos. Los criterios de corrección asegurarían que no existe ninguna sucesión de decisiones que un jugador podría tomar cuya consecuencia sea que éste no pueda completar el juego o que algunas áreas del mismo queden permanentemente inaccesibles.

Durante este trabajo se discutirán las herramientas actualmente disponibles, la motivación detrás del desarrollo de ésta herramienta, la definición del propio proyecto, su diseño como producto software, su desarrollo, las decisiones tomadas durante cada una de éstas fases, las conclusiones, los resultados obtenidos y posibles mejoras o ampliaciones futuras que podrían aplicarse al producto.



\palabrasclave{aplicación de escritorio, videojuegos, diseño, interfaz gráfica, espacios jugables, niveles, instanciar, objetos lógicos, estructura lógica implícita, motor de videojuegos, teoría de grafos, inteligencia artificial}
