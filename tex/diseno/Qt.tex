Qt\cite{qt1}\cite{qt2}\cite{qtDoc} es el framework de C++ empleado para la implementación, tanto lógica como de interfaz de usuario, de esta aplicación.

Se han empleado diversas clases de Qt para la codificación de la aplicación, desde clases generales como QString, QList, etc. hasta clases más específicas de ciertos ámbitos como QRectF, QPolygonF, QPainterPath para las operaciones geométricas, QJsonObject, QJsonDocument para las operaciones de persistencia y generación de ficheros .yy y distintas clases que heredan de QWidget y QLayout para la construcción de la interfaz gráfica de usuario.
También se ha hecho uso del sistema de slots y señales que proporciona Qt para la comunicación entre las distintas componentes de la aplicación y para la interacción con el usuario. Este sistema sustituye a los callbacks usuales en el desarrollo de aplicaciones de escritorio.