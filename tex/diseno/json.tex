JSON\cite{json} es un estándar\footnote{The JSON Data Interchange Syntax: \url{http://www.ecma-international.org/publications/files/ECMA-ST/ECMA-404.pdf}} de formato de descripción de objetos para el intercambio de datos.
se ha empleado este estándar en los módulos de persistencia, para guardar los datos en ficheros de manera que luego puedan ser leídos por el software para reconstruir los datos de una sesión anterior.
\textit{Game Maker Studio 2} también utiliza este formato para mantener, en ficheros separados, la persistencia de los proyectos desarrollados con dicho motor, por lo que el módulo de esta aplicación que se encarga de exportar datos del diseño, debe hacer uso del estándar JSON.

Adicionalmente, debido a cómo se generan los datos, el formato JSON debidamente escrito en ficheros es amigable tanto para los desarrolladores, que pueden leer ficheros escritos en dicho formato, como con tecnologías como git para el control de versiones.