\begin{functional}
	\item \textbf{Dibujar \textit{espacios jugables}}\newline
		El usuario, en el contexto de \textit{diseño} debe poder dibujar \textit{espacios jugables} en el canvas haciendo click en un punto del canvas y soltando el click en otro.
		\begin{functional}
			\item \textbf{Anexión de \textit{espacios jugables}}\newline
				Si el conjunto \textbf{A} de los \textit{espacios jugables} que intersecan con el recién creado es no vacío, entonces se eliminarán los \textit{espacios jugables} pertenecientes a \textbf{A} y se creará un nuevo \textit{espacio jugable} resultante de la unión de todos los contenidos en \textbf{A} junto con el recién creado.
			\item \textbf{Indexación de \textit{espacios jugables}}\newline
				Todo \textit{espacio jugable} debe ser indexado por una estructura para su futura transformación en grafo.
			\item \textbf{Vértices de un \textit{espacio jugable}}\newline
				Para todo \textit{espacio jugable} deben definirse sus vértices en el canvas.
			\item \textbf{Grid de coordenadas}
				Los vértices de todo \textit{espacio jugable} deben pertenecer a un \textit{grid} de coordenadas.
		\end{functional}
	\item \textbf{Modificación de los \textit{espacios jugables}}\newline
		El usuario debe poder modificar la posición de las aristas y los vértices de un \textit{espacio jugable} preexistente.
	\item \textbf{Eliminación de \textit{espacios jugables}}\newline
		El usuario debe poder extraer areas rectangulares de un \textit{espacio jugable}, de manera que el resultado respete la definición de \textit{espacio jugable}.
	\item \textbf{Creación de vértices}\newline
		El usuario debe poder crear un nuevo vértice en una arista.
	\item \textbf{Eliminación de vértices}\newline
		El usuario debe poder eliminar un vértice siempre y cuando sus dos vértices vecinos estén alineados.
	\item \textbf{Crear un elemento bloqueador}\newline
		El usuario debe poder crear un elemento bloqueador dibujando una línea en el canvas.
		\begin{functional}
			\item \textbf{Especificación del tipo de elemento bloqueador}\newline
				El elemento bloqueador creado por defecto será de tipo bidireccional, pero el usuario debe poder especificar que sea de tipo unidireccional, en cuyo caso debe especificar en qué dirección tendrá efecto el bloqueo.
			\item \textbf{Cálculo de vecindad al crear un elemento bloqueador}\newline
				En el caso de que la creación de un elemento bloqueador divida un \textit{espacio jugable} en dos, se han de indexar estos dos nuevos \textit{espacios jugables} y se ha de definir su vecindad teniendo en cuenta el tipo de elemento bloqueador y sus condiciones de apertura.
			\item \textbf{Eliminación de elementos bloqueadores inválidos}\newline
				Si el elemento bloqueador que se desea crear no cumple con la definición de elemento bloqueador, este será eliminado del diseño.
		\end{functional}
	\item \textbf{Seleccionar un elemento bloqueador}\newline
		El usuario debe poder seleccionar un elemento bloqueador preexistente del canvas.
	\item \textbf{Eliminar un elemento bloqueador}\newline
		El usuario debe poder eliminar un elemento bloqueador preexistente del canvas.
	\item \textbf{Crear un elemento llave}\newline
		El usuario debe poder crear un elemento llave, instanciando un tipo de llave preexistente, en el canvas.
	\item \textbf{Seleccionar un elemento llave}\newline
		El usuario debe poder seleccionar un elemento llave preexistente del canvas.
	\item \textbf{Desplazar un elemento llave}\newline
		El usuario debe poder desplazar un elemento llave preexistente dentro del propio canvas.
	\item \textbf{Eliminar un elemento llave}\newline
		El usuario debe poder eliminar un elemento llave preexistente del canvas.
	\item \textbf{Zoom}\newline
		El usuario debe poder acercar y alejar la vista que tiene del canvas.
	\item \textbf{Desplazar vista}\newline
		El usuario debe poder desplazar la vista que tiene del canvas.
\end{functional}