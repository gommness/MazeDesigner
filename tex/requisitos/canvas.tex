%%\begin{functional}
	\item \textbf{Dibujar \textit{espacios jugables}}\newline
		El usuario, en el contexto de \textit{diseño} debe poder dibujar \textit{espacios jugables} en el canvas haciendo click en un punto del canvas y soltando el click en otro.
		\begin{functional}
			\item \textbf{Anexión de \textit{espacios jugables}}\newline
				Si el conjunto de los \textit{espacios jugables} que intersecan con el recién creado es no vacío, entonces se realizará la unión de los \textit{espacios jugables} preexistentes y el nuevo \textit{espacio jugable}
			\item \textbf{Indexación de \textit{espacios jugables}}\newline
				Todo \textit{espacio jugable} debe ser indexado por una estructura para su futura transformación en grafo.
			\item \textbf{Vértices de un \textit{espacio jugable}}\newline
				Para todo \textit{espacio jugable} deben definirse sus vértices en el canvas.
			\item \textbf{Grid de coordenadas}
				Los vértices de todo \textit{espacio jugable} deben pertenecer a un \textit{grid} de coordenadas.
		\end{functional}
	\item \textbf{Eliminación de \textit{espacios jugables}}\newline
		El usuario debe poder extraer áreas rectangulares de un \textit{espacio jugable}, de manera que el resultado respete la definición de \textit{espacio jugable}.
	\item \textbf{Crear un elemento bloqueador}\newline
		El usuario debe poder crear un elemento bloqueador dibujando una línea en el canvas.
		\begin{functional}
			\item \textbf{Especificación del tipo de elemento bloqueador}\newline
				El elemento bloqueador creado por defecto será de tipo bidireccional, con ambas condiciones siendo la condición vacía.
			\item \textbf{Cálculo de vecindad al crear un elemento bloqueador}\newline
				En el caso de que la creación de un elemento bloqueador divida un \textit{espacio jugable} en dos, se han de indexar estos dos nuevos \textit{espacios jugables} y se ha de definir su vecindad teniendo en cuenta el tipo de elemento bloqueador y sus condiciones de apertura.
		\end{functional}
	\item \textbf{Seleccionar un elemento bloqueador}\newline
		El usuario debe poder seleccionar un elemento bloqueador preexistente del canvas.
	\item \textbf{Eliminar un elemento bloqueador}\newline
		El usuario debe poder eliminar un elemento bloqueador preexistente del canvas.
	\item \textbf{Crear un elemento llave}\newline
		El usuario debe poder crear un elemento llave, instanciando un tipo de llave en el canvas.
	\item \textbf{Seleccionar un elemento llave}\newline
		El usuario debe poder seleccionar una instancia de llave del canvas.
	\item \textbf{Eliminar un elemento llave}\newline
		El usuario debe poder eliminar una instancia de llave del canvas.
	\item \textbf{Zoom}\newline
		El usuario debe poder acercar y alejar la vista que tiene del canvas.
	\item \textbf{Desplazar vista}\newline
		El usuario debe poder desplazar la vista que tiene del canvas.
%%\end{functional}