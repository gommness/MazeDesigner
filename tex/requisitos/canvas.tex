%%\begin{functional}
	\item \textbf{Dibujar \dfnpl{espacio-jugable}.}\newline
		El usuario, en el contexto de diseño, debe poder dibujar \dfnpl{espacio-jugable} en el \textit{canvas} haciendo \textit{click} en un punto del \textit{canvas} y soltando el \textit{click} en otro.
		\begin{functional}
			\item \textbf{Anexión de \dfnpl{espacio-jugable}.}\newline
				Si el conjunto de los \dfnpl{espacio-jugable} que intersecan con el recién creado es no vacío, entonces se realizará la unión de los \dfnpl{espacio-jugable} preexistentes y el nuevo \dfn{espacio-jugable}
			\item \textbf{Indexación de \dfnpl{espacio-jugable}.}\newline
				Todo \dfn{espacio-jugable} debe ser indexado por una estructura para su futura transformación en grafo.
			\item \textbf{Grid de coordenadas.}\newline
				Los vértices de todo polígono que conforme un \dfn{espacio-jugable} deben estar ajustados a una rejilla de coordenadas.
		\end{functional}
	\item \textbf{Eliminación de \dfnpl{espacio-jugable}.}\newline
		El usuario debe poder extraer áreas rectangulares de un \dfn{espacio-jugable}, de manera que el resultado sea un polígono con ángulos de 90 grados o de 270 grados.
	\item \textbf{Crear un elemento bloqueador.}\newline
		El usuario debe poder crear un elemento bloqueador dibujando una línea en el \textit{canvas}.
		\begin{functional}
			\item \textbf{Condición de apertura.}
				Todo elemento bloqueador debe tener asignada una condición de apertura. Por defecto, la condición vacía.
			\item \textbf{Localización válida para elementos bloqueadores.}\newline
				El elemento bloqueador creado debe estar contenido en su totalidad dentro de los espacios jugables.
			\item \textbf{Direcciones permitidas para elementos  bloqueadores.}\newline
				El elemento bloqueador creado solo puede tener una dirección totalmente vertical o totalmente horizontal.
			\item \textbf{Cálculo de vecindad al crear un elemento bloqueador.}\newline
				A la hora de realizar una exploración, en el caso de que la creación de un elemento bloqueador divida un \dfn{espacio-jugable} en dos, se han de indexar estos dos nuevos \dfnpl{espacio-jugable} y se ha de definir su vecindad teniendo en cuenta el tipo de elemento bloqueador y sus condiciones de apertura.
		\end{functional}
	\item \textbf{Seleccionar un elemento bloqueador.}\newline
		El usuario debe poder seleccionar un elemento bloqueador preexistente del \textit{canvas}.
	\item \textbf{Eliminar un elemento bloqueador.}\newline
		El usuario debe poder eliminar un elemento bloqueador preexistente del \textit{canvas}.
	\item \textbf{Crear un elemento llave.}\newline
		El usuario debe poder crear un elemento llave, instanciando un modelo de llave en el \textit{canvas}.
	\item \textbf{Seleccionar un elemento llave.}\newline
		El usuario debe poder seleccionar una instancia de llave del \textit{canvas}.
	\item \textbf{Eliminar un elemento llave.}\newline
		El usuario debe poder eliminar una instancia de llave del \textit{canvas}.
	\item \textbf{Zoom.}\newline
		El usuario debe poder acercar y alejar la vista que tiene del \textit{canvas}.
	\item \textbf{Desplazar vista.}\newline
		El usuario debe poder desplazar la vista que tiene del \textit{canvas}.
%%\end{functional}