%%\begin{functional}
	\item \textbf{Crear tipos de llaves}\newline
		El usuario debe poder crear modelos de llaves especificando su nombre y si es un \dfn{power-up} o no.
	\item \textbf{Eliminar tipos de llaves}\newline
		El usuario debe poder eliminar modelos de llaves previamente creados.
	\item \textbf{Modificar el nombre de un tipo de llave}\newline
		El usuario debe poder modificar el nombre que se le hubiera asignado a un modelo de llave.
	\item \textbf{Modificar la clase de un tipo de llave}\newline
		El usuario debe poder modificar la clase de un modelo de llave, es decir, modificar si es o no un \dfn{power-up}.
	\item \textbf{Seleccionar un tipo de llave}\newline
		El usuario debe poder seleccionar un modelo de llave de entre la lista de todos los tipos de llaves.
	\item \textbf{Unicidad en los nombres de las llaves}\newline
		Dos modelos de llaves no podrán compartir nombre. Si el usuario intenta crear una llave con el mismo nombre que otra, se cambiará el nombre, concatenando un número al final del mismo, para que no coincidan.
	\item \textbf{Caracteres permitidos en los nombres de las llaves}\newline
		Los nombres de las llaves solo podrán estar compuestos de por caracteres alfanuméricos, junto con los caracteres ``-'', ``\textunderscore'' y ``.''. Si el usuario intenta crear una llave conteniendo otros caracteres, no se efectuará el cambio de nombre. 
	\item \textbf{Nombres de llaves no permitidos}\newline
		Las llaves no podrán ser nombradas solamente con números ni usando la palabra clave ``have'' por pertenecer al lenguaje declarativo de las condiciones.
%%\end{functional}