La aplicación será una herramienta de diseño de niveles, por tanto abarcará las tareas de construcción de espacios, la instanciación solamente en el diseño de agentes lógicos como llaves y puertas y la posterior exploración del diseño para la comprobación de que éste es correcto, es decir, no existe una cadena de acciones o exploración que tiene como consecuencia que el jugador no pueda acceder a algún área del juego.
De la misma manera, la aplicación deberá poder exportar estos diseños a un proyecto de \textit{Game Maker Studio 2}. Esta exportación creará tantos ficheros .yy como habitaciones hayan sido definidas en el diseño, y al contenido de dichos ficheros .yy, solo se exportarán los objetos sólidos que conforman las paredes, suelos y techos de la habitación en cuestión. Esto es, las instancias lógicas antes mencionadas que se encuentren dentro de una habitación no serán exportadas al fichero .yy dado que estas instancias lógicas, en la implementación del juego, serán objetos cuyo funcionamiento deberá implementar el desarrollador dentro del propio motor, ajustándose a los requisitos del juego como producto final.
Es decir, esta herramienta no es un \dfn{editor-de-niveles}, pues no cubre la funcionalidad de colocar instancias en los espacios de juego.

Además de esto, a la hora de realizar la exploración, la herramienta solamente tendrá en cuenta la vecindad de componentes conexas del diseño a través de las instancias lógicas de las puertas. Es decir, la herramienta no tendrá en cuenta mecánicas de juego como ``el jugador no puede saltar más de una altura X y por tanto no puede llegar a determinados lugares'' a la hora de determinar qué áreas son accesibles o no. Para modelar algo así, el desarrollador debería partir el espacio de juego con una puerta cuya condición de apertura sea ``poder saltar más de una altura X'', es decir, ``el jugador debe poseer un \dfn{power-up} que permitiría saltar la altura X''.