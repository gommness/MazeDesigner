La aplicación será una herramienta de diseño de niveles, por tanto abarcará las tareas de construcción de espacios, la instanciación (solamente en el diseño) de agentes lógicos como llaves y puertas; y la posterior exploración del diseño para la comprobación de que éste es correcto, es decir, no existe una cadena de acciones o exploración que tiene como consecuencia que el jugador no pueda acceder a algún área del juego.

De la misma manera, la aplicación deberá poder exportar estos diseños a un proyecto de \textit{GameMaker Studio 2}. Esta exportación creará tantos ficheros \textit{.yy} (utilizados por el motor para representar los \dfnpl{nivel}) como habitaciones hayan sido definidas en el diseño. Al contenido de dichos ficheros \textit{.yy} solo se exportarán los objetos sólidos que conforman las paredes, suelos y techos de la habitación en cuestión.
Es decir, las instancias lógicas antes mencionadas no serán exportadas al fichero \textit{.yy} dado que en la implementación del juego serán objetos cuyo funcionamiento deberá implementar el desarrollador.

Esta herramienta no es un \dfn{editor-de-niveles}, pues no cubre la funcionalidad de colocar instancias en los espacios de juego.

Además de esto, a la hora de realizar la exploración, solamente se tienen en cuenta las componente conexas del diseño funcionando como nodos y las puertas funcionando como aristas. Es decir, la herramienta no tendrá en cuenta mecánicas de juego como ``el jugador no puede saltar más de una altura X y por tanto no puede llegar a determinados lugares'' a la hora de determinar qué áreas son accesibles o no. Modelar algo así en esta herramienta es tarea del desarrollador.