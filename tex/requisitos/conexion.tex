%%\begin{functional}
	\item \textbf{Creación de pantallas}\newline
		El usuario debe poder seleccionar regiones rectangulares del canvas que representarán las pantallas del juego que está diseñando. Al crearse, las pantallas tendrán un nombre asignado por defecto.
	\item \textbf{Seleccionar una pantalla}\newline
		El usuario debe poder seleccionar una pantalla.
	\item \textbf{Cambiar el nombre de una pantalla}\newline
		El usuario debe poder cambiar el nombre de una pantalla.
	\item \textbf{Eliminar una pantalla}\newline
		El usuario debe poder eliminar un región que representa una pantalla de su juego
	\item \textbf{Conversión a ficheros .yy.room}\newline
		El sistema debe ser capaz de transformar las pantallas a ficheros .yy.room de \textit{Game Maker Studio 2}, con las dimensiones y nombre determinados por el diseño del usuario. En estos ficheros figurarán solamente los bloques sólidos que cubrirán los \textit{espacios no jugables} del diseño que haya dentro de cada una de las pantallas.
		\begin{functional}
			\item \textbf{Construcción de un área con el mínimo número de rectángulos}\newline
				Los \textit{espacios no jugables}, que tendrán forma poligonal cuyos ángulos serán múltiplos de 90 grados, han de cubrirse del menor número de rectángulos posible. Estos rectángulos, en la traducción a ficheros .yy.room, se representarán como bloques sólidos del juego.
		\end{functional}
%%\end{functional}