\item \textbf{Edición de condiciones de apertura.}\newline
	Cuando el usuario haya seleccionado un elemento bloqueador, debe ser capaz de describir las condiciones bajo las cuales este se abrirá.
	\begin{functional}
		\item \textbf{Sintaxis del lenguaje.}\newline
			Las condiciones deben seguir una sintaxis determinada (ver apéndice \ref{CAP:APCONDICIONES})
		\item \textbf{Condición vacía.}\newline
			La condición vacía del lenguaje se interpreta como una satisfacible siempre.
		\item \textbf{Condiciones no satisfacibles}\newline
			Deben poder crearse condiciones no satisfacibles.
		\item \textbf{Orden de precedencia.}\newline
			A la hora de evaluar una expresión del lenguaje, se debe respetar el orden de precedencia habitual: resolución del contenido de los paréntesis primero y conexión de izquierda a derecha.
	\end{functional}
\item \textbf{Sintetización de condiciones.}\newline
	El sistema debe ser capaz de sintetizar las condiciones escritas por el usuario para ser asignadas a otros elementos del sistema.
\item \textbf{Listas de costes}\newline
	Una condición debe poder calcular la lista o listas de objetos llave requeridos para ser satisfecha.