\item \textbf{Edición de condiciones de apertura}\newline
	Cuando el usuario haya seleccionado un elemento bloqueador, debe ser capaz de describir las condiciones bajo las cuales este se abrirá. Dicha descripción debe ser escrita siguiendo la sintaxis de un lenguaje.
	\begin{functional}
		\item \textbf{Condición vacía}\newline
			El lenguaje para la descripción de condiciones debe admitir a condición vacía.
		\item \textbf{Condición simple}\newline
			El lenguaje para la descripción de condiciones debe admitir condiciones simples, compuestas por un único requisito para la apertura
		\item \textbf{Operadores lógicos}\newline
			El lenguaje para la descripción de condiciones debe reconocer conectores formados por operadores lógicos
		\item \textbf{Condiciones compuestas}\newline
			El lenguaje para la descripción de condiciones debe admitir condiciones compuestas, formadas conectando condiciones (simples o compuestas) con operadores lógicos
		\item \textbf{Orden de precedencia}\newline
			El lenguaje para la descripción de condiciones debe respetar el orden de precedencia habitual: resolución del contenido de los paréntesis primero y conexión de izquierda a derecha.
	\end{functional}
\item \textbf{Sintetización de condiciones}\newline
	El sistema debe ser capaz de sintetizar las condiciones escritas por el usuario, siguiendo la sintaxis del lenguaje de descripción de condiciones, en clases compatibles con las aristas de los grafos, para inducir en estos una exploración condicionada.