Además del tradicional índice de contenidos se disponen de otros muchos índices controlables a través de las opciones de la clase. Si se utiliza la opción \textbf{loall} se presentarán todos los índices. Por el contrario la opción \textbf{lonone} elimina todas estas listas. Si no se usa ni estas opciones ni las que se describen a continuación se presentarán todos los índices. Las siguientes opciones no se excluyen entre si de tal forma que es posible indicar varias simultáneamente:

\begin{description}
  \item [loa] Presenta la lista de algoritmos.
  \item [loc] Presenta la lista de códigos.
  \item [loe] Presenta la lista de ecuaciones.
  \item [lof] Presenta la lista de figuras.
  \item [lot] Presenta la lista de tablas.
  \item [ltb] Presenta la lista de cuadros de texto.
\end{description}

Como norma de estilo se recomienda no añadir una lista si no tiene al menos 3 elementos aunque queda a discreción del autor.
