C++\cite{cpp}\cite{cpp2} es un lenguaje de programación multiparadigma cuyo origen es la extensión del lenguaje de programación C al paradigma de la \dfn{orientacion-a-objetos}. Tras esto, conforme C++ ha ido evolucionando, se han incorporado facultades propias de otros paradigmas, de manera que ahora engloba a la \dfn{programacion-estructurada}, a la \dfn{orientacion-a-objetos}, a la \dfn{programacion-generica} y en las versiones más actualizadas contiene ciertos conceptos de \dfn{programacion-funcional}.
Como características distintivas, C++ permite al programador el uso de punteros, gestión de la memoria y también permite la sobrecarga de operadores.
Actualmente, C++ se encuentra en su versión C++17, destacando la versión C++11 que realizó mejoras a nivel del núcleo del lenguaje.
C++ es un lenguaje compilado que destaca por su alto rendimiento, sus capacidades para paralelizar tareas y por la gestión de la memoria por parte del programador.

Junto a C++, se encuentra Qt\cite{qtDoc}\cite{qt1}\cite{qt2}, un framework que no pertenece al estándar pero de uso muy extendido para la tarea del desarrollo de aplicaciones de escritorio.
Qt incluye librerías y APIs escritas en C++ que abarca ámbitos desde renderización, cálculos geométricos, multithreading, control del input de dispositivos hardware y hasta realización de peticiones http.
Actualmente, Qt está en su versión 5.12 y ofrece un lenguaje propio para declaración de componentes gráficos (QML\cite{qml}) y un binding con el lenguaje de programación python.