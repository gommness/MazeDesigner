Históricamente, el rol del diseñador de este tipo de juegos comenzaba dibujando en papel el diseño de los mismos (tal y como observamos en la figura \ref{FIG:ARTE:SKETCH}). Se empezaba por un esquema en forma de grafo en el que se describía el progreso lógico del jugador (qué requisitos debe cumplir para acceder al área siguiente) y más tarde se implementaba esta lógica del progreso en los niveles diseñados ya a nivel visual.
Más adelante, se empezaron a utilizar herramientas software para sustituir al papel, pero las tareas permanecían divididas en un diseño lógico y visual, más cercano a lo que aparece en el producto final.
Además de esto, la traducción de estos diseños a componentes del juego se ha llevado a cabo manualmente en distintos entornos de desarrollo y, en la actualidad, se utilizan los motores de videojuegos para este propósito.

\begin{figure}{FIG:ARTE:SKETCH}{Sketch realizado por Ari Gibson. Es un boceto del mapa del videojuego  \textit{Hollow Knight (Team Cherry, 2017)} que se hizo en las frases tempranas del desarrollo.}
	\image{10cm}{}{imagenes/hollow-knight-sketch}
\end{figure}