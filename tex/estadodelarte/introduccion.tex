Históricamente, el rol del diseñador de este tipo de juegos comenzaba dibujando en papel el diseño de los mismos, se empezaba por un esquema en forma de grafo en el que se describía el progreso lógico del jugador (qué requisitos debe cumplir para acceder al área siguiente) y más tarde se implementaban esta lógica del progreso en los niveles diseñados ya a nivel visual.
Más adelante, se empezaron a utilizar herramientas software para sustituir el papel, pero las tareas permanecían divididas en un diseño lógico y visual, más cercano a lo que aparece en el producto final.
Además de esto, la traducción de estos diseños a componentes del juego, se ha llevado a cabo manualmente en distintos entornos de desarrollo y, en la actualidad, se utilizan los motores de videojuegos para este propósito.