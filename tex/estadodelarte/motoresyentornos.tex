En la actualidad, la producción de videojuegos se lleva a cabo en motores de videojuegos, que conforman un entorno de desarrollo. Algunos de estos motores son de propósito general y permiten el desarrollo de cualquier tipo de juego.
\begin{enumerate}
	\item \textbf{Unity} Es un motor de videojuegos creado por Unity technologies. Se trata de un motor de propósito general que permite la creación de videojuegos tanto en 3D como en 2D.
	\item \textbf{Unreal Engine} Es un motor de videojuegos creado por Epic Games. Es también un motor de propósito general para creación de juegos tanto en 3D como en 2D.
	\item \textbf{Game Maker Studio 2} Es un motor de videojuegos creado por YoYo Games. También de propósito general pero limitado a la creación de juegos en 2D.
\end{enumerate}

De entre los motores enumerados, escogeremos \textit{Game Maker Studio 2} por ser de propósito general y, al limitar su ámbito a los juegos en dos dimensiones, se simplifican muchas de las complicaciones que presentan los juegos en tres dimensiones, entre las que se encuentra la creación de escenarios, que es el propósito principal que abarca la herramienta Maze Designer.