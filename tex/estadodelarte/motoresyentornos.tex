En la actualidad la producción de videojuegos se lleva a cabo en motores de videojuegos, que conforman un entorno de desarrollo. Algunos de estos motores son de propósito general y permiten el desarrollo de cualquier tipo de juego.
\begin{enumerate}
	\item \textbf{Construct\footnote{Construct 3 - Game Making Software: \url{https://www.construct.net/en}}} Es un motor de videojuegos creado por \textit{Scirra Ltd}. Se trata de un motor de propósito general orientado a un público con pocos conocimientos técnicos en el desarrollo de videojuegos. Permite crear videojuegos en dos dimensiones.
	\item \textbf{framework XNA4\cite{xna4}} Es un framework desarrollado por \textit{Microsoft} para el desarrollo de videojuegos. Fue descontinuado en 2010
	\item \textbf{MonoGame\footnote{MonoGame - One framework for creating powerful cross-platform games: \url{http://www.monogame.net/}}} Es una implementación del framework XNA4 de Microsoft, desarrollada por \textit{MonoGame Team}. Se trata de una herramienta open-source que permite la creación de videojuegos tanto en dos como en tres dimensiones.
	\item \textbf{Unity\footnote{Unity for all: \url{https://unity.com}}} Es un motor de videojuegos creado por \textit{Unity technologies}. Se trata de un motor de propósito general que permite la creación de videojuegos tanto en tres como en dos dimensiones.
	\item \textbf{Unreal Engine\footnote{Unreal Engine - Make something unreal with the most powerful creation engine: \url{https://www.unrealengine.com/en-US/}}} Es un motor de videojuegos creado por \textit{Epic Games}. Es también un motor de propósito general para creación de juegos tanto en tres como en dos dimensiones.
	\item \textbf{GameMaker Studio 2\cite{gamemaker}} Es un motor de videojuegos creado por \textit{YoYo Games}. También de propósito general pero limitado a la creación de juegos en dos dimensiones.
\end{enumerate}

De entre los motores enumerados, se ha escogido \textit{GameMaker Studio 2} por ser de propósito general y suficientemente completo.
Al limitar su ámbito a los juegos en dos dimensiones, se simplifican muchas de las complicaciones que presentan los juegos en tres dimensiones, entre las que se encuentra la creación de escenarios, que es el propósito principal que abarca la herramienta \textit{Maze Designer}.