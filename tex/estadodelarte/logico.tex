Las herramientas del diseño lógico, se basan en la abstracción del progreso del jugador en forma de grafo.
Dado que el progreso se modela como un grafo dirigido, la tecnología orientada a cubrir estos requerimientos consiste en software de visualización y manipulación de grafos. Para estas tareas específicas, existen las siguientes herramientas:
\begin{enumerate}
	\item \textbf{graphviz} http://www.graphviz.org/ Se trata de una aplicación para diseñar y visualizar información abstracta estructurada en grafos.
	\item \textbf{librería diagrams de Haskell} http://hackage.haskell.org/package/diagrams Se trata de una librería de haskell con la que se pueden declarar y visualizar grafos. Es una potente librería de fácil utilización, sin embargo tiene el defecto ser necesario programar en haskell, lo cual es una tarea que puede escaparse de las competencias de un diseñador o puede llegar a ser demasiado complejo para la tarea en cuestión.
	\item \textbf{Finite State Machine Designer} http://madebyevan.com/fsm/ Se trata de una aplicación web para diseñar grafos de forma visual cuyo objetivo principal es el diseño de autómatas finitos. Sin embargo, debido a la naturaleza de grafo de los diseños realizados y a la sencillez de utilización, puede cumplir con los requisitos de esta categoría de diseño, aunque su naturaleza de aplicación web puede resultar incómoda en especial para diseños muy complejos.
\end{enumerate}
Todas las herramientas mencionadas comparten el defecto de ser herramientas puramente abstractas para representar el progreso en forma de grafo, sin preocuparse de la visualización de los niveles y, por tanto, la implementación de la lógica diseñada con estas herramientas a niveles de un producto es una tarea que el diseñador debe realizar manualmente.