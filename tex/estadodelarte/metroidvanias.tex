Los videojuegos de tipo mazmorra, englobando videojuegos que la industria ha bautizado como \textit{metroidvania} y \textit{zelda-like} entre otros, es un género de productos de ocio digital en los que el usuario controla a un avatar en un entorno que debe explorar, usualmente controlando la progresión del jugador a través de dicho entorno mediante puertas bloqueadas u obstáculos que debe sortear consiguiendo llaves o habilidades nuevas para su avatar.

El foco central de estos juegos es la exploración y la progresión a través del escenario, a diferencia de otros géneros de videojuegos, donde el foco puede estar centrado en la trama, en asumir el rol de un determinado personaje o en simular una experiencia como puede ser en un juego de deporte. El diseño de cómo es la exploración depende del desarrollador y de la visión que tenga sobre su producto.

Al hablar de la exploración de estos juegos, se suelen emplear términos como \dfn{exploracion-lineal}, \dfn{exploracion-ramificada} y \dfn{backtracking} para describirlos. Durante la exploración de los escenarios del juego, el jugador puede recolectar diversos objetos que, dependiendo del juego, tienen un uso u otro, sin embargo se repiten mucho los patrones de \dfn{objeto-consumible} y \dfn{power-up} ya que estos en particular pueden utilizarse para el control del progreso del jugador.

Algunos de los títulos de éste género más notables de la industria se presentaron en la figura \ref{FIG:INTRO:JUEGOS}.