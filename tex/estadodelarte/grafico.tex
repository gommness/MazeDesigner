Las herramientas de diseño gráfico, que no son capaces de realizar ningún tipo de abstracción lógica, sino que son empleadas únicamente para representar el aspecto del diseño, con el nivel de detalle que el usuario decida.
La actual tecnología empleada para este tipo de diseño, varía desde el diseño en papel, hasta herramientas software como:
\begin{enumerate}
	\item \textbf{Photoshop\footnote{Photoshop main page: \url{https://www.photoshop.com/}}} La herramienta de diseño gráfico profesional de \textit{Adobe}, permite realizar detalladas manipulaciones de imágenes para diseñar el espacio de juego e incluso permite el uso de capas para mayor utilidad.
	\item \textbf{Gimp\footnote{GIMP - The Free and Open Source Image Editor: \url{https://www.gimp.org/}}} La herramienta open source de diseño gráfico que cumple, a los efectos del diseño de niveles de un videojuego, los mismos propósitos que \textit{photoshop}
	\item \textbf{Paint.net\footnote{Paint dot net main page: \url{https://www.getpaint.net/}}} La herramienta gratuita de edición de imágenes que, a pesar de ser sencilla, es bastante potente en cuanto a las funcionalidades que ofrece, alcanzando estándares similares a \textit{GIMP}.
	\item \textbf{Tiled\footnote{Tiled - your free, easy to use and flexible level editor: \url{https://www.mapeditor.org/}}} Es un editor de mapas orientado específicamente al diseño de videojuegos, pero su función es únicamente la colocación de elementos gráficos para la construcción de los niveles.
\end{enumerate}
Todas las herramientas mencionadas presentan el problema en común de que no aplican ningún tipo de abstracción lógica al diseño, de manera que la comprobación de que un éste sea correcto queda a discreción del usuario.
Además de esto, la integración de los diseños con un motor de videojuegos en específico, a excepción de \textit{Tiled} es íntegramente manual.